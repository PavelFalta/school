\section{Struktura vstupních dat}
\label{sec:data_structure}

Vstupní data pro náš systém jsou uložena v souboru CSV a obsahují informace o klíčových bodech, jejich predikovaných pozicích a vahách. V této sekci popíšeme strukturu těchto dat a způsob, jakým jsou organizována.

\subsection{Formát dat}

Datový soubor \texttt{data-recovery.csv} obsahuje následující typy sloupců:

\begin{itemize}
    \item \textbf{target\_kpX\_y, target\_kpX\_x} - skutečné (cílové) souřadnice klíčového bodu X
    \item \textbf{pred\_kpX\_posY\_y, pred\_kpX\_posY\_x} - predikované souřadnice klíčového bodu X, predikce Y (0-4)
    \item \textbf{pred\_kpX\_valY} - váha přiřazená predikci Y pro klíčový bod X
    \item \textbf{pred\_kpX\_centroid\_y, pred\_kpX\_centroid\_x} - souřadnice centroidu (váženého průměru) pro klíčový bod X
    \item \textbf{pred\_kpX\_sigma\_y, pred\_kpX\_sigma\_x} - standardní odchylka predikcí pro klíčový bod X
\end{itemize}

kde X je identifikátor klíčového bodu (0, 1, 2, ...) a Y je index predikce (0, 1, 2, 3, 4).

\subsection{Příklad struktury dat}

Pro lepší pochopení struktury dat uvádíme zjednodušený příklad několika záznamů z datasetu (pouze pro bod kp0):

\begin{table}[H]
\centering
\begin{tabular}{lcccccc}
\toprule
\textbf{Sloupec} & \textbf{Řádek 1} & \textbf{Řádek 2} & \textbf{Řádek 3} \\
\midrule
target\_kp0\_y & 45.2 & 48.7 & 42.3 \\
target\_kp0\_x & 102.5 & 105.1 & 99.8 \\
pred\_kp0\_pos0\_y & 46.1 & 49.2 & 41.5 \\
pred\_kp0\_pos0\_x & 103.2 & 106.3 & 98.7 \\
pred\_kp0\_val0 & 0.85 & 0.92 & 0.78 \\
pred\_kp0\_pos1\_y & 44.8 & 47.6 & 42.9 \\
pred\_kp0\_pos1\_x & 101.9 & 104.8 & 100.5 \\
pred\_kp0\_val1 & 0.75 & 0.67 & 0.82 \\
... & ... & ... & ... \\
pred\_kp0\_centroid\_y & 45.6 & 48.9 & 42.4 \\
pred\_kp0\_centroid\_x & 102.8 & 105.4 & 99.6 \\
pred\_kp0\_sigma\_y & 0.72 & 0.85 & 0.62 \\
pred\_kp0\_sigma\_x & 0.68 & 0.79 & 0.73 \\
\bottomrule
\end{tabular}
\caption{Ukázka struktury dat pro klíčový bod 0 (kp0)}
\label{tab:data_example}
\end{table}

\subsection{Význam jednotlivých hodnot}

\begin{itemize}
    \item \textbf{Cílové souřadnice (target)} - skutečné pozice klíčových bodů, které chceme predikovat. Tyto hodnoty slouží jako ground truth pro vyhodnocení přesnosti našich predikcí.
    
    \item \textbf{Predikované pozice (pred\_kpX\_posY)} - pět různých predikcí pozice pro každý klíčový bod. Tyto hodnoty mohou pocházet z různých zdrojů nebo metod, každá s jinou úrovní spolehlivosti.
    
    \item \textbf{Váhy predikcí (pred\_kpX\_valY)} - hodnoty spolehlivosti pro každou predikci. Vyšší hodnota znamená vyšší důvěru v danou predikci. Tyto váhy jsou využity v metodě bodu s největší vahou i jako příznaky pro model Random Forest.
    
    \item \textbf{Centroid (pred\_kpX\_centroid)} - vážený průměr všech predikcí, kde váhy odpovídají hodnotám spolehlivosti. Centroid představuje "průměrnou" predikci, která bere v úvahu všechny dostupné informace.
    
    \item \textbf{Sigma (pred\_kpX\_sigma)} - standardní odchylka predikcí. Tato hodnota vyjadřuje rozptyl predikcí a může indikovat nejistotu v určení pozice klíčového bodu.
\end{itemize}

\subsection{Příprava dat pro trénování}

Před trénováním modelů Random Forest je potřeba připravit data následujícím způsobem:

\begin{enumerate}
    \item \textbf{Identifikace klíčových bodů} - pomocí regulárního výrazu najdeme všechny klíčové body v datasetu
    \item \textbf{Rozdělení dat} - rozdělíme dataset na trénovací a testovací část v poměru 80:20
    \item \textbf{Extrakce příznaků} - pro každý klíčový bod vytvoříme příznakový vektor obsahující všechny dostupné informace
    \item \textbf{Škálování příznaků} - pomocí StandardScaler normalizujeme hodnoty příznaků pro lepší konvergenci modelu
\end{enumerate}

Tato příprava dat zajišťuje, že modely Random Forest mají k dispozici všechny relevantní informace v vhodném formátu pro trénování. 